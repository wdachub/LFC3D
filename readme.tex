\documentclass[twoside,a4paper]{article}
\usepackage{geometry}
\geometry{margin=3cm, vmargin={0pt,3cm}}
\setlength{\topmargin}{-1cm}
\setlength{\paperheight}{29.7cm}
\setlength{\textheight}{24.3cm}

% useful packages.




\usepackage{amsfonts}
\usepackage{amsmath}
\usepackage{amssymb}
\usepackage{amsthm}
\usepackage{enumerate}

\usepackage{graphicx} 
\usepackage{float} 
\usepackage{subfigure} 
\usepackage{caption} 


\usepackage{multicol}
\usepackage{fancyhdr}
\usepackage{layout}


\begin{document}

\pagestyle{fancy}
\fancyhead{}
\lhead{ }
\chead{ }
\rhead{}


\section{About this package}

This matlab package implements the three-dimensional Lagrangian flux calculation  method described in the manuscript ``Lagrangian Flux Calculation Through A Time-Dependent Surface For Scalar Conservation Laws'' by  L. Ding \& B. Huang \& S. Hu \& Q. Zhang. This document lists the relevant files for reproducing the tables and figures in that manuscript.

\section{File}

\subsection{../MoveLFC/src}
This folder contains the main files of our Lagrangian flux calculation method.

\begin{enumerate}
\item  \texttt{fluxDR3D.m} is the main subroutine of this package, which constructs a spline-approximated generating surface and computes the flux of a scalar function through a moving surface by our LFC method.

\item \texttt{DRIntegral.m} is the subroutine that evaluates the integral over the spline approximated generating surface.

\item \texttt{splinegauss.m} generates the quadrature rule for evaluating the integral over a spline approximated region in 3-dimensional space. 


\item \texttt{cubature\_manager.m} provides nodes and weights of a quadrature routine on $[-1,1]$, which is modified from the package  maintained by  Alvise Sommariva and Marco Vianello \cite{sommariva2009gauss}   \verb|http://www.math.unipd.it/~marcov/software.html|.

\item \texttt{RungeKutta.m} is the subroutine that solves ordinary differential equation by explicit Runge-Kutta method.


%  The codes are described by L.N. Trefethen in his clenshaw-curtis paper and by Waldvogel (published by Bit).

  

\item \texttt{flowmap.m} is the subroutine for computing the trajectory of the Lagrangian particle. 
  
\end{enumerate}



\subsection{../MoveLFC/test}
This folder contains the files which reproduce the tables and figures in the manuscript.

\begin{enumerate}
\item \texttt{table1.m, table2\_1.m, table2\_2.m,table3\_1.m,table3\_2.m} compute the errors and convergence rates of each tests, which respectively reproduce Table 1,2,3 in the manuscript. The Latex output files are saved in \texttt{../MoveLFC/Tables}. 

\item \texttt{plot2a.m,plot2b.m,plot2c.m,plot2d.m,plotFigure3.m,plotFigure4.m}  reproduce data for Fig 2, 3, 4 in the manuscript, respectively. The output data files are saved in 

\texttt{../MoveLFC/Figures/Data}
\end{enumerate}
\subsection{../MoveLFC/Figures}
\subsubsection{../MoveLFC/Figures/plotFigure2}
This folder contains the files which reproduce the figure2 in the manuscript. To get figure2, you should run \texttt{../MoveLFC/test/plot2a.m, plot2b.m, plot2c.m, plot2d.m} to generate data files in \texttt{../MoveLFC/Figures/Data}.
\begin{enumerate}
    \item \texttt{plotDRcomponent.m} is the subroutine that plots each component surfaces for the donating region.
    \item \texttt{PaperPlotDR.m} is the subroutine that plots the whole donating region.
    \item \texttt{Figure2.m} generates figure2 in the manuscript.
\end{enumerate}
\subsubsection{../MoveLFC/Figures/plotFigure3\&4}
This folder contains the files which reproduce the figure3 and figure4 in the manuscript. To get figure3, you should run \texttt{../MoveLFC/test/plotFigure3} to generate data files in \texttt{../MoveLFC/Figures/Data}; to get figure4, you should run \texttt{../MoveLFC/test/plotFigure4} to generate data files in \texttt{../MoveLFC/Figures/Data}.
\begin{enumerate}
    \item \texttt{plotDRcomponent.m} is the subroutine that plots each component surfaces for the donating region.
    \item \texttt{PaperPlotDR.m} is the subroutine that plots the whole donating region.
    \item \texttt{Figure3.m} generates figure3 in the manuscript.
    \item \texttt{Figure4.m} generates figure4 in the manuscript.
\end{enumerate}

\subsection{../useCase}
This folder includes the moving surface, scalar functions, and velocity fields used in reproducing the tables or figures in our manuscript.



\bibliography{bib/FiniteVolume,bib/quadrature,bib/MyPaper}   % name your BibTeX data base
\bibliographystyle{abbrv}

\end{document}



%%% Local Variables: 
%%% mode: latex
%%% TeX-master: t
%%% End: 
